\documentclass[nofootinbib,amssymb,amsmath]{revtex4}
\usepackage{mathtools}
\usepackage{amsthm}
\usepackage{algorithm}
\usepackage{algpseudocode}
\usepackage{lmodern}
\usepackage{graphicx}
\usepackage{color}
\usepackage{bm}

%Put an averaged random variable between brackets
\newcommand{\ave}[1]{\left\langle #1 \right\rangle}

\newcommand{\vzero}{{\bf 0}}
\newcommand{\vI}{{\bf I}}
\newcommand{\vb}{{\bf b}}
\newcommand{\vd}{{\bf d}}
\newcommand{\vf}{{\bf f}}
\newcommand{\vc}{{\bf c}}
\newcommand{\vv}{{\bf v}}
\newcommand{\vz}{{\bf z}}
\newcommand{\vn}{{\bf n}}
\newcommand{\vm}{{\bf m}}
\newcommand{\vG}{{\bf G}}
\newcommand{\vQ}{{\bf Q}}
\newcommand{\vM}{{\bf M}}
\newcommand{\vW}{{\bf W}}
\newcommand{\vX}{{\bf X}}
\newcommand{\vPsi}{{\bf \Psi}}
\newcommand{\vSigma}{{\bf \Sigma}}
\newcommand{\vlambda}{{\bf \lambda}}
\newcommand{\vpi}{{\bf \pi}}
\newcommand{\valpha}{{\bm{\alpha}}}
\newcommand{\vbeta}{{\bm{\beta}}}
\newcommand{\vomega}{{\bm{\omega}}}
\newcommand{\vLambda}{{\bf \Lambda}}
\newcommand{\vA}{{\bf A}}

\newcommand{\code}[1]{\texttt{#1}}

\newtheorem{lemma}{Lemma}
\newtheorem{corollary}{Corollary}

\def\SL#1{{\color [rgb]{0,0,0.8} [SL: #1]}}
\def\DB#1{{\color [rgb]{0,0.8,0} [DB: #1]}}

\newcommand{\HOM}{$\mathsf{Hom}$}
\newcommand{\HET}{$\mathsf{Het}$}
\newcommand{\REF}{$\mathsf{Ref}$}
\newcommand{\epss}{\varepsilon}

\begin{document}

\title{Germline Filtering in Mutect}
\author{David Benjamin}
\email{davidben@broadinstitute.org}
\affiliation{Broad Institute, 75 Ames Street, Cambridge, MA 02142}

\date{\today}

\maketitle

\section{Germline Filter}\label{germline-filter}
Suppose we have detected an allele such that its (somatic) likelihood in the tumor is $\ell_t$ and its (diploid) likelihood in the normal is $\ell_n$.  By convention, both of these are relative to a likelihood of $1$ for the allele \textit{not} to be found.  If we have no matched normal, $\ell_n = 1$.  Suppose we also have the population allele frequency $f$ of this allele.  Then the prior probability for the normal to be heterozygous or homozygous alt for the allele is $2f(1-f) + f^2$ and the prior probability for the normal genotype not to contain the allele is $(1-f)^2$.  Finally, suppose that the prior for this allele to arise as a somatic variant is $\pi$.

We can determine the posterior probability that the variant exists in the normal genotype by calculating the unnormalized probabilities of four possibilities:
\begin{enumerate}
\item The variant exists is both the normal and the tumor samples.  This has unnormalized probability $\left(2f(1-f) + f^2 \right) \ell_n \ell_t (1 - \pi)$.
\item The variant exists in the tumor but not the normal.  This has unnormalized probability $(1-f)^2 \ell_t \pi$.
\item The variant exists in neither the tumor nor the normal.  This has unnormalized probability $(1-f)^2 (1 - \pi)$.
\item The variants exists in the normal but not the tumor.  This is biologically very unlikely.  Furthermore, if it \textit{did} occur we wouldn't care about filtering the variant as a germline event because we wouldn't call it as a somatic event.  Thus we neglect this possibility.
\end{enumerate}

Normalizing, we obtain the following posterior probability that an allele is a germline variant:
\begin{equation}
P({\rm germline}) = \frac{(1)}{(1) + (2) + (3)} = \frac{\left(2f(1-f) + f^2 \right) \ell_n \ell_t (1 - \pi)}{\left(2f(1-f) + f^2 \right) \ell_n \ell_t (1 - \pi) + (1-f)^2 \ell_t \pi + (1-f)^2 (1 - \pi)}.
\end{equation}

To filter, we set a threshold on this posterior probability.

\end{document}